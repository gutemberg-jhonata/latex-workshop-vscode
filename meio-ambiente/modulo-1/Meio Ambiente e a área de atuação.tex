\documentclass{article}

\title{
    Meio Ambiente e a Área de Atuação \\
    \large Engenharia da Computação
}
\author{
    Jhonata Gutemberg
}
\date{Março 2021}

\begin{document}
    \maketitle

    \par Com o surgimento da tecnologia da informação (TI) empresas e
    instituições foram beneficiadas através da virtualização da informação,
    automatizando e facilitando processos que antes eram completamente 
    manuais através de tecnologias como e-mail eletrônico, certificação
    digital, mídias de informação e internet.
    \par Entretanto essas novas técnologias trouxeram novas necessidades,
    uma vêz que os recursos de computação agridem o meio ambiente, uma
    vez que trazem o aumento do consumo da energia elétrica, acarretando
    em um aumento necessário de produção de energia, afetando desta forma
    o meio ambiente, já que ou métodos mais utilizados de produção desta
    energia agridem consideravelmente o meio ambiente. Além deste fator
    o mau gerenciamento do uso destes recursos e a produção destes recursos
    causam muita agressão ao ser humano e meio ambiente.
    \par Devido aos efeitos colaterais trazidos pelos recursos de TI, 
    surge o movimento Green IT ou TI Verde. Tendo por objetivo erradicar
    ou diminuir os danos causados ao meio ambiente. 

\end{document}}