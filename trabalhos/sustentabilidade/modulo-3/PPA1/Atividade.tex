\documentclass[a4paper, 12pt]{article}
\usepackage{fontspec}

\setmainfont{Arial}
\tolerance=10000
\hyphenpenalty=10000
\renewcommand{\baselinestretch}{1.5}

\title{
    Atividade 1: Módulo Energia \\
    \large Engenharia da Computação
}
\author{
    Jhonata Gutemberg
}
\date{Abril 2021}

\begin{document}
    \maketitle
    
    \par Os principais desafios da gestão de energia são: irregularidade das fontes de energia, valor dos investimentos e riscos ecológicos.
    
    \par Muitas vezes as fontes de energias renováveis não tem alta disponibilidade. Nem sempre há uma incidência interessante de raios solares ou ventos generosos para a captação desta energia. Para solucionar este problema existem os sistemas híbridos, entretendo não são viáveis dependendo da região, que pode ter uma incidência de raios solares interessante, porém não é propicio para a instalação de uma usina hidroelétrica ou eólica.

    \par Mesmo a implementação de energias renováveis tendo diversas vantagens seu custo de instalação ainda é muito elevado pois necessitam de equipamentos altamente tecnológicos.

    \par Outro grande desafio é minimizar o impacto ambiental causado pelas usinas hidroelétricas que por sua vez precisam alagar uma região para serem instaladas causando problemas como: realocação da comunidade local, desmatamento e alteração do curso do rio.
\end{document}}