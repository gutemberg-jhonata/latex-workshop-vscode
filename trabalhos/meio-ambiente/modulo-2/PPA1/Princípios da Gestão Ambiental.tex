\documentclass[a4paper, 12pt]{article}
\usepackage{fontspec}
\usepackage{indentfirst}

\setmainfont{Arial}
\tolerance=10000
\hyphenpenalty=10000
\renewcommand{\baselinestretch}{1.5}

\title{
    Princípios da Gestão Ambiental \\
    \large Engenharia da Computação
}
\author{
    Jhonata Gutemberg \\
    Karine Fernanda
}
\date{Abril 2021}

\begin{document}
    \maketitle

    \section{Sistema de gestão ambiental}

    \par O sistema de gestão ambiental é um conjunto de procedimentos de gestão e administração organizacional, que tem como objetivo obter o melhor relacionamento com o meio ambiente. Esse sistema é estabelecido pela norma ISO 14001, sendo esta uma organização internacional de padronização.

    \par A fim de implementar o sistema de gestão ambiental dentro da organização é necessário a utilização dos 5 princípios da gestão ambiental, sendo eles respectivamente: política ambiental; planejamento; implementação e operação; verificação e ação corretiva; análise crítica.

    \subsection{Política ambiental}

    \par É o principal documento elaborado pela organização, na qual expõem suas intenções e princípios em relação ao seu desempenho ambiental global, que estabelece uma estrutura para a ação e definição dos seus objetivos e metas ambientais.

    \subsection{Planejamento}

    \par No planejamento deve incluir os seguintes tópicos: aspectos ambientais, requisitos legais e outros requisitos, objetivos e metas; e programas de gestão ambiental.

    \subsection{Implementação e operação}

    \par Este princípio consiste na implementação da norma ISO sendo necessário atender o que está previsto em sua política, metas e objetivos por meio da efetivação de algumas estruturas, sendo estas: estrutura organizacional e responsabilidade; treinamento, conscientização e competência; comunicação; documentação do sistema de gestão ambiental; controle de documentos; controle operacional e preparação e atendimento a emergências.
    
    \subsection{Verificação e ação corretiva}

    \par O objetivo deste princípio é criar condições para verificar se a empresa está de acordo com o programa de gestão ambiental previamente definido, trata as medidas preventivas, identifica aspectos não desejáveis e mitiga quaisquer impactos negativos. A verificação e ação corretiva são orientadas por quatro etapas do processo de gestão ambiental: monitoramento e medição; não-conformidade e ações corretivas e preventivas; registros e auditoria do sistema de gestão ambiental.

    \subsection{Análise crítica}

    \par É o momento em que a administração deve identificar a necessidade de possíveis alterações na política ambiental, nos seus objetivos e metas, ou em outros itens do sistema.

    \section{Exemplo da aplicação do SGA}

    \par Segundo Salton (2013), o sistema de gestão ambiental pode ser aplicados em empresas de mineração seguindo os princípios da gestão ambiental.

    \subsection{Política ambiental}
    
    \par Salton diz que para se obter a implementação do sistema de gerenciamento ambiental a organização deve declarar:

    \begin{itemize}
        \item Cumprir com a legislação aplicável ao Sistema de Gestão Ambiental e aos requisitos relacionados aos aspectos e impactos ambientais.
        \item Buscar a melhoria contínua da qualidade ambiental, por meio do envolvimento de seus colaboradores na conscientização ambiental com ações no processo educativo, despertando o interesse pelo meio ambiente que incentive à
        reciclagem e evite o desperdício dos recursos naturais.
        \item Capacitar os colaboradores com relação à gestão do meio ambiente e educação ambiental, para que possa haver o entendimento dos mesmos sobre seus papéis e a importância de suas atividades.
        \item Identificar e buscar a prevenção da degradação ambiental decorrente das operações de lavra e beneficiamento, com ênfase na minimização da geração
        de resíduos sólidos e redução do consumo de água e energia.
        \item Prevenir a poluição buscando, sempre que possível, a eliminação na fonte geradora, a redução ou o controle de seus aspectos ambientais, priorizando os resíduos sólidos industriais.
    \end{itemize}

    \subsection{Planejamento}
    
    \par Planejamento das atividades realizado entre a equipe ambiental, os diretores e os líderes de cada área, onde foi definido o cronograma das atividades e os responsáveis pelas ações.

    \subsection{Implementação e operação}

    \par Com o planejamento das ações necessárias se iniciou a execução das atividades, de acordo com o cronograma previamente definido.

    \subsection{Verificação e ação corretiva}

    \par O monitoramento foi realizado para verificar os resultados das ações ambientais. Exemplos: monitoramento da qualidade da água, monitoramento de geração de ruídos, monitoramento dos efluentes.

    \subsection{Análise crítica}

    \par Nessa etapa foram discutidos os resultados, analisando o que precisa ser melhorado.

    \section{Referências}
    
    \noindent SALTON, Camylla. Aplicação do sistema de gestão ambiental em empresas: Estudo de Caso de uma Mineradora de Basalto em Ibiporã/PR. Universidade Tecnológica Federal do Paraná (UTFPR). Londrina 2013.
    \\
    \noindent Disponível em: http://repositorio.roca.utfpr.edu.br
    \\
    \noindent Acesso em: 11 de abril de 2021.
\end{document}