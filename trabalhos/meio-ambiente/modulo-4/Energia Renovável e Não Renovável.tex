\documentclass[a4paper, 12pt]{article}
\usepackage{fontspec}
\usepackage{indentfirst}

\setmainfont{Arial}
\tolerance=10000
\hyphenpenalty=10000
\renewcommand{\baselinestretch}{1.5}

\title{
    Energia Renovável e Não Renovável \\
    \large Engenharia da Computação
}
\author{
    Jhonata Gutemberg \\
    Karine Fernanda
}
\date{Maio 2021}

\begin{document}
    \maketitle
    \par A energia hidrelétrica é a energia obtida através do aproveitamento do potencial hidráulico de um rio. A geração desta energia se dá através de usinas instaladas em rios. A força da água do rio passa pelas tubulações da usina, realizando o movimento das turbinas que são conectadas a um gerador, desta forma a usina converte energia potencial em energia elétrica. Por ser gerada á partir da energia potencial dos rios este tipo de energia é considerada renovável, devido á própria característica renovável dos rios.

    \par A energia nuclear é obtida através da fissão do átomo de urânio enriquecido, este processo de fissão por sua vez libera uma grande quantidade de energia. Neste processo de fissão, ocorre a geração de material radioativo, entretanto a energia gerada é limpa. Este processo de geração de energia é considerada não renovável, pois depende do consumo do material urânio, que tem quantidade limitada na natureza.
\end{document}}