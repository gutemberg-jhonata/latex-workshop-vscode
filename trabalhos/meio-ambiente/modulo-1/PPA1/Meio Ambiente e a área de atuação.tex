\documentclass[a4paper, 12pt]{article}
\usepackage{fontspec}

\setmainfont{Arial}
\tolerance=10000
\hyphenpenalty=10000
\renewcommand{\baselinestretch}{1.5}

\title{
    Meio Ambiente e a Área de Atuação \\
    \large Engenharia da Computação
}
\author{
    Jhonata Gutemberg
}
\date{Março 2021}

\begin{document}
    \maketitle

    \par Com o surgimento da tecnologia da informação (TI) empresas e
    instituições foram beneficiadas através da virtualização da informação,
    automatizando e facilitando processos que antes eram completamente 
    manuais através de tecnologias como e-mail eletrônico, certificação
    digital, mídias de informação e internet.
    
    \par Entretanto essas novas tecnologias trouxeram novas necessidades,
    uma vez que os recursos de computação agridem o meio ambiente, trazendo o aumento do consumo da energia elétrica, acarretando
    em um aumento necessário de produção de energia, afetando desta forma o meio ambiente. 
    
    \par Os métodos mais utilizados de produção desta energia agridem consideravelmente o meio ambiente. Além deste fator, o mau gerenciamento do uso destes recursos e a produção destes recursos causam muita agressão ao ser humano e meio ambiente.
    
    \par Devido aos efeitos colaterais trazidos pelos recursos de TI, 
    surge o movimento Green IT ou TI Verde. Tendo por objetivo erradicar
    ou diminuir os danos causados ao meio ambiente.
    
    \par O TI Verde pode ser implementado através das abordagens: 
    incremental, estratégica e radical verde. A abordagem incremental
    consiste em preservar a infraestrutura atual de TI, incorporando 
    políticas e medidas simples, a fim de atingir objetivos pequenos, 
    tais medidas costumam ser de fácil implantação e sem muito custo,
    tendo retorno quase imediato e pode ser observado, analisando a redução
    no consumo da energia elétrica. 
    
    \par Já a abordagem estratégica realiza uma auditoria na infraestrutura de TI. Desta forma todos os equipamentos são analisados de maneira individual ou em grupos que podem ser formados por tipos de equipamentos ou divididos por aplicação, buscando assim a melhor relação custo beneficio e diminuição na geração de CO2.
    
    \par Por fim a abordagem radical verde agrega a implementação da abordagem estratégica com a implementação de uma política de compensação de carbono por neutralizar a emissão de gases que geram o efeito estufa.

    \par Desta forma é possível perceber uma mudança na consciência por parte das organizações empresariais, públicas e universitárias principalmente. Também que existe uma tendência por parte destas empresas em aumentar a aplicabilidade da TI Verde.

    \par O TI Verde tem obtido avanços e aceitação por parte da sociedade que é o essencial. Muita coisa ainda existe para ser feita, pois este é só um começo. Em breve será mais perceptível os benefícios da TI Verde tanto na sociedade, quanto na economia e no meio ambiente.

\end{document}}