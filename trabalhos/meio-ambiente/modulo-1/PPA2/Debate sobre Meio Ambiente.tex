\documentclass[a4paper, 12pt]{article}
\usepackage{fontspec}

\setmainfont{Arial}
\tolerance=10000
\hyphenpenalty=10000
\renewcommand{\baselinestretch}{1.5}

\title{
    Debate sobre Meio Ambiente \\
    \large Engenharia da Computação
}
\author{
    Jhonata Gutemberg
}
\date{Março 2021}

\begin{document}
    \maketitle

    \noindent\fbox{
        \begin{minipage}{\dimexpr\textwidth-2\fboxsep-2\fboxrule\relax}
            A questão ecológica ou ambiental deve se restringir à preservação dos ambientes naturais intocados e ao combate da poluição; as demais questões — envolvendo saneamento, saúde, cultura, decisões sobre políticas de energia, de transportes, de educação, ou de desenvolvimento — são extrapolações que não devem ser da alçada dos ambientalistas.
        \end{minipage}
    } \\

    \par As questões ambientais não devem se restringir somente à preservação e ao combate a poluição, tendo em vista que o meio ambiente é muito mais amplo. As pessoas, empresas, cidades e etc. Tudo faz parte do meio ambiente, desta forma as questões ambientais devem envolver outras questões como saneamento, cultura, educação e política. \\

    \noindent\fbox{
        \begin{minipage}{\dimexpr\textwidth-2\fboxsep-2\fboxrule\relax}
            Os que defendem o meio ambiente são pessoas radicais e privilegiadas, não necessitam trabalhar para sobreviver, mantêm-se alienadas da realidade das exigências impostas pela necessidade de desenvolvimento; defendem posições que só perturbam quem realmente produz e deseja levar o país para um nível melhor de desenvolvimento.
        \end{minipage}
    } \\

    \par Devido a nossa cultura, muitos acreditam que somente é possível preservar o meio ambiente em detrimento ao desenvolvimento. Porém essa afirmação não é verdadeira uma vez que o surgimento de novas tecnologias podem reduzir os custos de produção e o impacto ambiental, obtendo a eco-eficiência. A preservação do meio ambiente também se preocupa com questões econômicas, a fim de alcançar um ambiente sustentável. \\

    \noindent\fbox{
        \begin{minipage}{\dimexpr\textwidth-2\fboxsep-2\fboxrule\relax}
            É um luxo e um despropósito defender, por exemplo, animais ameaçados de extinção, enquanto milhares de crianças morrem de fome ou de diarréia na periferia das grandes cidades, no
            Norte ou no Nordeste.
        \end{minipage}
    } \\
    
    \par A fome e desnutrição é o problema mais importante que um país pode enfrentar e deve ser totalmente priorizado. Porém esse problema não excluí os demais, os animais em extinção devem ser protegidos, uma vez que não têm uma correlação direta com o problema anterior, mais ainda sim tem sua relevância. Dizer que não devemos defender animais em extinção enquanto milhares de crianças morrem de fome é o mesmo que dizer que não devemos nos preocupar com a poluição, enquanto inúmeras pessoas não te um lar. Ambos os problemas têm seu grau de importância e devem ser resolvidos. \\

    \noindent\fbox{
        \begin{minipage}{\dimexpr\textwidth-2\fboxsep-2\fboxrule\relax}
            Quem trabalha com questões relativas ao meio ambiente pensa de modo romântico, ingênuo, acredita que a natureza humana é intrinsecamente “boa” e não percebe que antes de tudo vem a dura realidade das necessidades econômicas. Afinal, a pior poluição é a pobreza, e para haver progresso é normal algo ser destruído ou poluído.
        \end{minipage}
    } \\
    
    \par Trabalhar com o questões relativas ao meio ambiente é trabalhar de forma realista e não ingênua, uma vez são considerados fatores como impacto ambiental e as consequências desse impacto para nós e para as gerações futuras. A afirmação de que somente é possível haver  progresso destruindo a natureza é falsa, pois é possível obter o progresso de forma sustentável, a poluição e destruição da natureza só ocorrem devido a tentativa de obter lucro a todo custo e despreocupação com estas questões. \\

    \noindent\fbox{
        \begin{minipage}{\dimexpr\textwidth-2\fboxsep-2\fboxrule\relax}
            Idealiza-se a natureza, quando se fala da “harmonia da natureza”. Como se pode falar em “harmonia”, se na natureza os animais se atacam violentamente e se devoram? Que harmonia é essa?
        \end{minipage}
    } \\

    \par A morte faz parte do ecossistema, animais precisam alimentar-se de outros animais ou plantas para sobreviverem, a própria extinção das espécies é muitas vezes causada pela própria natureza, como consequência da seleção natural. As ações da natureza não são boas ou ruins, elas simplesmente acontecem. A ação de desequilíbrio da natureza é então causada pelas ações do ser humano, quando este consome muito mais do que precisa para sobreviver.

\end{document}