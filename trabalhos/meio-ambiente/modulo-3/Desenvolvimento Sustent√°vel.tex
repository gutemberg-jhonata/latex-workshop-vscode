\documentclass[a4paper, 12pt]{article}
\usepackage{fontspec}
\usepackage{indentfirst}

\setmainfont{Arial}
\tolerance=1000
\hyphenpenalty=1000
\renewcommand{\baselinestretch}{1.5}

\title{
    Desenvolvimento Sustentável \\
    \large Engenharia da Computação
}
\author{
    Jhonata Gutemberg
    \\ Karine Fernanda
}
\date{Maio 2021}

\begin{document}
    \maketitle
    \par A definição do conceito de desenvolvimento sustentável iniciou em 1987 através da publicação do relatório Brundtland. O relatório de Brundtland define o desenvolvimento sustentável como aquele que atende as necessidades do presente sem comprometer as possibilidades de as gerações futuras atenderem suas próprias necessidades. Desta forma o relatório apresenta os conceitos de necessidade e limitações.
    \par O conceito de necessidades está associado às necessidades essenciais dos pobres do mundo enquanto que a ideia de limitações está relacionada ao estado da tecnologia e organização social na capacidade do meio ambiente de atender às necessidades atuais e futuras.
    \par Em 1992, durante a Conferência das Nações Unidas sobre o Meio Ambiente e o Desenvolvimento, a ECO-92, foram elaborados alguns documentos importantes que 
    reforçavam a necessidade de um desenvolvimento mais sustentável para o planeta. Dentre eles, destacam-se a Declaração do Rio sobre Meio Ambiente e Desenvolvimento e a Agenda 21.
    \par Nessa conferência, o termo Desenvolvimento Sustentável foi definitivamente popularizado pela ampla cobertura da mídia à Conferência. No entanto, naquela época, o uso indiscriminado deste termo era criticado, sobretudo nos discursos governamentais e nos projetos de desenvolvimento.
    \par Temia-se que pudesse se tornar mais um dos modismos incorporados pelas elites, sobretudo do terceiro mundo, uma “maquiagem de velhos discursos com uma coloração “verde”.
    \par No ano 1997, Elkington sugeriu que a atividade corporativa deveria ser orientada por 
    três dimensões: pela lógica do desenvolvimento (dimensão econômica), ser socialmente justa (dimensão social) e ser ambientalmente correta (dimensão ambiental). Esse modelo ficou conhecido como Triple Bottom Line (Profits, People, Planet). Seu modelo propunha que as organizações visassem, além do desempenho econômico, também as dimensões ambientais e sociais. Sendo recepcionado nos documentos oficiais da Conferência das Nações Unidas em Joanesburgo em 2002.
    \par Na agenda da Sustentabilidade, tida para harmonizar o pilar financeiro com o 
    pensamento emergente do pilar ambiental, tornou-se ainda mais complicada do que os 
    executivos haviam imaginado. Atualmente, se pensa nos termos de um dos três pilares,com enfoque na prosperidade econômica, na qualidade ambiental e na justiça social Além disso, “entende necessidades baseadas de uma perspectiva ocidental globalizante que ignora as inúmeras diferenças culturais entre as nações.
    \par Portanto, o atual modelo de sociedade está baseado no ‘desenvolvimento 
    sustentável’ da economia e, efetivamente, não ultrapassou o sentido de desenvolvimento econômico; e este não está suportando a pressão exercida pela crise ambiental de âmbito global.
\end{document}}